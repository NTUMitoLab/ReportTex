% !TeX root = ./main.tex


%行高
\renewcommand{\baselinestretch}{1.5} 

%版型
\usepackage[a4paper,left=3cm,right=2cm,top=2.5cm,bottom=2.5cm]{geometry}

%數學
\usepackage{amsmath}

%表格
\usepackage{tabularx}
\usepackage{ltablex}

%中文設定
\usepackage{xeCJK} %讓中英文字體分開設置
% Noto Font: Download here: https://www.google.com/get/noto/#sans-hant
\setCJKmainfont{Noto Sans CJK TC}

\usepackage{booktabs} % 畫分隔線

%隱藏標題碼
\makeatletter
\renewcommand\@seccntformat[1]{}
\makeatother

%顏色
\usepackage[dvipsnames]{xcolor}

%標頭文字. 上欄
\usepackage{fancyhdr} 
\pagestyle{fancy}
\fancyhf{}
%頁碼
\cfoot{\thepage} 

%超連結
\usepackage[unicode]{hyperref} %用 unicode防止中文亂碼
\usepackage{url}

\hypersetup{
    colorlinks=true,
    linkcolor=NavyBlue,
    filecolor=NavyBlue,      
    urlcolor=NavyBlue,
    citecolor=NavyBlue
} %超連結顏色

% 可愛圖
\usepackage{fontawesome5} %gallery of fontawesome: https://mirror-hk.koddos.net/CTAN/fonts/fontawesome5/doc/fontawesome5.pdf

%文包圖
\usepackage{wrapfig}
\usepackage{subcaption}
\usepackage{stfloats} %caption footnote
%彩色方塊文字
\usepackage{tcolorbox}

%圖的文字編碼
\usepackage[labelfont=bf, font={small}]{caption}

%中文package
\usepackage{CJKutf8}

%畫圖
\usepackage{tikz}
\usetikzlibrary{positioning}

%Reference
\usepackage[numbers]{natbib}
%\usepackage[round]{natbib}
\bibliographystyle{unsrtnat}

\usepackage{multicol}
