% !TeX root = ../main.tex

\begin{tcolorbox}[title=隨機 SEIR 模型]
SEIR 模型的疾病擴散模型。以下方程式描述的是每個時間點~($t$) 從一個狀態轉換到下一個階段的人數以 $N$表示(狀態的分類在Fig. \ref{fig:seir})。由於每個人轉換狀態的機率為 Bernoulli 機率分佈,群體下的機率則可以用下面的方程式表示:

\begin{align} 
    N^{(t)}_{S_{c_i}\rightarrow E_{c_i}} & \sim  Binom(S^{(t)}_{c_1},  \lambda^{(t)}_{c_1}) \label{eq:N_S2E}\\
    N^{(t)}_{E_{c_i}\rightarrow 1_{c_i}} & \sim Binom(E^{(t)}_{c_i}, 1/\delta_{E}) \\
    N^{(t)}_{1_{c_1}\rightarrow R_{c_i}} & \sim Binom(I^{(t)}_{c_i}, 1/ \delta_{i})  
\end{align}

$N^{t}_{i\rightarrow j}$ 指的是在時間$t$時由$i$狀態轉換至$j$狀態的人數。遵從 $Binomial$ 分佈, PMF (Probability mass functoin) 為~$N \sim Binomial(n,p) = {N\choose n} p^{n}(1-p)^{N-n} $。其中 P 代表的是一個人轉換狀態的機率. 而 $\delta_E$ 和 $\delta_i$ 分別為平均潛伏期(96hr)和平均發病期(84hr)\cite{vinfo2020}。
\end{tcolorbox}
