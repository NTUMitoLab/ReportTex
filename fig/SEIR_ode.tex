% !TeX root = ../main.tex


\begin{figure}[h]
\centering
\begin{tikzpicture}[ every node/.style={circle,draw=black}]
  
    \node (s) at (0,0) {$S$};
    \node (e) [right=of s] {$E$};
    \node (i1) [right=of e] {$I_1$};
    \node (i2) [right=of i1] {$I_2$};
    
    \node (r) [right=of i2] {$R$};
    
    
    \path[->] (s) edge (e);
    \path[->] (e) edge   (i1);
    \path[->] (i1) edge   (i2);
    \path[->] (i2) edge (r);
     
\end{tikzpicture}

\caption{\citeauthor{modus2020} 決定性 SEIR 模型. 當中分成五種族群: 未感染的族群(Susceptible, S), 潛伏期的族群(Exposed, E), 發病且有感染力的族群(Symptomatically infected, $I_1$), 無症狀感染者(Asymptomatically infected, $I_2$) 以及 復原的族群(Recovered, R).}
\label{fig:seir-ode}
\end{figure}