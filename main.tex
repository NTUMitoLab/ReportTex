% !TEX encoding = UTF-8 Unicode


\documentclass[12pt, a4paper]{article}





% !TeX root = ./main.tex


%行高
\renewcommand{\baselinestretch}{1.5} 

%版型
\usepackage[a4paper,left=3cm,right=2cm,top=2.5cm,bottom=2.5cm]{geometry}

%數學
\usepackage{amsmath}

%表格
\usepackage{tabularx}
\usepackage{ltablex}

%中文設定
\usepackage{xeCJK} %讓中英文字體分開設置
% Noto Font: Download here: https://www.google.com/get/noto/#sans-hant
\setCJKmainfont{Noto Sans CJK TC}

\usepackage{booktabs} % 畫分隔線

%隱藏標題碼
\makeatletter
\renewcommand\@seccntformat[1]{}
\makeatother

%顏色
\usepackage[dvipsnames]{xcolor}

%標頭文字
\usepackage{fancyhdr} 


%超連結
\usepackage[unicode]{hyperref} %用 unicode防止中文亂碼
\usepackage{url}

\hypersetup{
    colorlinks=true,
    linkcolor=NavyBlue,
    filecolor=NavyBlue,      
    urlcolor=NavyBlue,
    citecolor=NavyBlue
} %超連結顏色

%文包圖
\usepackage{wrapfig}
\usepackage{subcaption}
\usepackage{stfloats} %caption footnote
%彩色方塊文字
\usepackage{tcolorbox}

%圖的文字
\usepackage[labelfont=bf, font={small}]{caption}

%中文package
\usepackage{CJKutf8}

%畫圖
\usepackage{tikz}
\usetikzlibrary{positioning}

%Reference
\usepackage[numbers]{natbib}
\bibliographystyle{unsrtnat}

\usepackage{multicol}


\pagestyle{fancy}
\fancyhf{}
\rhead{
研究計畫
}
\lhead{
博士班申請-臺大生醫電子與資訊學研究所甲組
}

\cfoot{\thepage}



\title{\textbf{研究計畫}}

\author{邱紹庭(\href{mailto:r07945001@ntu.edu.tw}{r07945001@ntu.edu.tw})\thanks{現臺大生醫電資所電子組碩士三年級,於\href{https://ntubmse.com/}{魏安祺老師實驗室}研究生物系統 (\href{https://drive.google.com/file/d/1dyBDxwvc3T4O5-BkM7Hrb8Lq1OcsbrCi/view?usp=sharing}{個人履歷}).}}


\begin{document}

\maketitle

\section{讀書計畫}

我在魏安祺老師實驗室研究期間,主要研究項目為生物系統領域。用微分方程系統和控制系統理論研究粒線體往細胞核傳訊的過程。碩士班的研究放在數學建模與控制系統分析上\cite{chiu2020}

\section{研究進程}


%https://mermaid-js.github.io/mermaid-live-editor/#/edit/eyJjb2RlIjoiZ2FudHRcbiAgIFxuICAgIGRhdGVGb3JtYXQgIFlZWVktTU0tRERcbiAgICB0aXRsZSBSZXNlYXJjaCBTY2hlZHVsZVxuXG4gICAgc2VjdGlvbiDnoJTnqbblhaflrrlcbiAgICDlrbjnv5IgQWdlbnRzLmpsIOS4puaetuioreeglOeptiBwYWNrYWdlOiAyMDIxLTA3LTAxLCAyMDIxLTA4LTAxXG4gICAgXG4gICAgXG4gICAgc2VjdGlvbiDlj7DlpKfljZrlo6tcbiAgICDnoqnoq5bmlofmipXmkJ7lnIvpmpvmnJ_liIo6IDIwMjEtMDMtMjIsIDIwMjEtMDYtMjdcbiAgICDmj5Dlh7rnoJTnqbZwcm9wb3NhbDogMjAyMS0xMi0yMCwgMjAyMS0xMi0yNVxuICAgIOS_ruiqsijljZrkuInkuIopOiAyMDIzLTA5LTE1LDIwMjQtMDEtMTVcbiAgICDkv67oqrIo5Y2a5LiJ5LiLKTogMjAyNC0wMi0yNSwyMDI0LTA2LTMwXG4gICAg5Y2a5aOr5a245L2N6ICDOiAyMDI0LTA3LTIyLCAyMDI0LTA3LTIzXG4gICAgXG5cbiAgICBzZWN0aW9uIOmZveS6pOeJmemGq1xuICAgIOWtuOagoeiqsueoiyjkupTkuIopOiAgZG9uZSwgMjAyMS0wOS0xNCwgMjAyMi0wMS0xN1xuICAgIOWtuOagoeiqsueoiyjkupTkuIopOiAgZG9uZSwyMDIyLTAyLTI1LCAyMDIyLTA1LTMwXG4gICAg6Ieo5bqK5a-m57-SOiBkb25lLDIwMjItMDYtMDEsIDIwMjMtMDUtMzBcbiAgICDkuozpmo7lnIvogIM6IGRvbmUsIDIwMjMtMDctMjMsMjAyMy0wNy0yNFxuXG4gICAgdG9kYXlNYXJrZXIgb2ZmXG4gICAgXG5cbiAgICAiLCJtZXJtYWlkIjp7InRoZW1lIjoiZGVmYXVsdCJ9LCJ1cGRhdGVFZGl0b3IiOmZhbHNlfQ



\begin{tcolorbox}[title=隨機 SEIR 模型]
    SEIR 模型的疾病擴散模型。以下方程式描述的是每個時間點~($t$) 從一個狀態轉換到下一個階段的人數以 $N$表示(狀態的分類在Fig. \ref{fig:seir})。由於每個人轉換狀態的機率為 Bernoulli 機率分佈,群體下的機率則可以用下面的方程式表示:
    
    \begin{align} 
        N^{(t)}_{S_{c_i}\rightarrow E_{c_i}} & \sim  Binom(S^{(t)}_{c_1},  \lambda^{(t)}_{c_1}) \label{eq:N_S2E}\\
        N^{(t)}_{E_{c_i}\rightarrow 1_{c_i}} & \sim Binom(E^{(t)}_{c_i}, 1/\delta_{E}) \\
        N^{(t)}_{1_{c_1}\rightarrow R_{c_i}} & \sim Binom(I^{(t)}_{c_i}, 1/ \delta_{i})  
    \end{align}
    
    $N^{t}_{i\rightarrow j}$ 指的是在時間$t$時由$i$狀態轉換至$j$狀態的人數。遵從 $Binomial$ 分佈, PMF (Probability mass functoin) 為~$N \sim Binomial(n,p) = {N\choose n} p^{n}(1-p)^{N-n} $。其中 P 代表的是一個人轉換狀態的機率. 而 $\delta_E$ 和 $\delta_i$ 分別為平均潛伏期(96hr)和平均發病期(84hr)\cite{vinfo2020}。
    \end{tcolorbox}


\bibliography{library}




\end{document}